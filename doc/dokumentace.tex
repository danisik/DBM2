\documentclass[12pt, a4paper]{article}
\usepackage[utf8]{inputenc}
%\usepackage[IL2]{fontenc}
\usepackage[czech]{babel}
\usepackage[pdftex]{graphicx}
\usepackage{mathtools}
\usepackage{amsmath}
\usepackage{svg}
\usepackage{textcomp}
\usepackage{listings,xcolor}
\usepackage[final]{pdfpages}
\usepackage{verbatim}
\usepackage{todonotes}
\usepackage[T1]{fontenc}
\usepackage{multirow}

\usepackage[nottoc,notlot,notlof]{tocbibind}
\usepackage[pdftex,hypertexnames=false]{hyperref}
\hypersetup{colorlinks=true,
  unicode=true,
  linkcolor=black,
  citecolor=black,
  urlcolor=black,
  bookmarksopen=true}

\newcommand{\floor}[1]{\lfloor #1 \rfloor}

\title{\textbf{Dokumentace semestrální práce} \\KIV/DBM2}
\author{Vojtěch Danišík}
\begin{document}

\begin{titlepage} 
	\newcommand{\HRule}{\rule{\linewidth}{0.5mm}} 
	\begin{center}
	\includegraphics[width=12cm]{img/fav_logo}\\
	\end{center}
	\textsc{\LARGE Západočeská univerzita v Plzni}\\[1.5cm] 	
	\textsc{\Large Databázové systémy a metody zprac.inf.2}\\[0.5cm] 
	\textsc{\large KIV/DBM2}\\[0.5cm] 
	\HRule\\[0.4cm]
	{\huge\bfseries Dokumentace semestrální práce}\\[0.4cm] 
	\HRule\\[1.5cm]

	\begin{minipage}{0.4\textwidth}
		\begin{flushleft}
			\large
			Vojtěch \textsc{Danišík}\newline
			A19N0028P\newline
			danisik@students.zcu.cz
		\end{flushleft}
	\end{minipage}
	\vfill\vfill\vfill
	\begin{flushright}
	{\large\today}
	\end{flushright}
	\vfill 
\end{titlepage}
\newpage
\tableofcontents
\newpage

\section{Zadání}
Již několik let je oblíbeným buzzwordem Open Data, což ve své podstatě jsou data poskytnutá veřejnosti pro libovolné další zpracování. Na našem území je nejčastěji s tímto pojmem spojován portál \href{https://data.gov.cz/}{https://data.gov.cz/}, který slouží jako katalog datasetů veřejné správy. Problém je, že momentálně katalog obsahuje 135 510 záznamů a je obtížné udělat si přehled o tom, jaké informace lze vůbec očekávat, že by se zde mohly objevit.

\begin{itemize}
\item Vyberte si podmnožinu dat na Portálu otevřených dat, případně jiný zdroj. \\
\item Popište makroskopicky témata, kterých se datasety týkají.\\
\item Vybranou podmnožinu zkoumejte detailně a popište jejich kvalitu* a použitelnost**.\\
\end{itemize}

\noindent * např. úplnost, smysluplné hodnoty, homogenní datový typ v rámci atributu, použití číselníků místo volného textu (problematické by byla existence výrazů "pozit.", "pozitivní", "kladné" ve stejném kontextu), ...\\

\noindent ** např. definice atributů, existence cross-referencí na jiné datasety, použití číselníků (kódová značka pro město místo jeho názvu), ...\\

\noindent Data: \href{https://data.gov.cz/}{https://data.gov.cz/}, nebo jiné dle uvážení.\\
 \newpage

\section{Analýza datasetu}
Z portálu https://data.gov.cz jsem vybral dataset s názvem: \textbf{Počty studentů krajských vyšších odborných škol - Královéhradecký kraj}.\\\\
\noindent Informace o datasetu:
\begin{itemize}
\item velikost souboru: 132 kB 
\item počet sloupců: 39 
\item počet řádků: 326 
\item rok: Školní rok 2019-2020 \\
\end{itemize}

\noindent Sloupce v datasetu by se dali rozdělit do 5 informativních bloků:
\begin{itemize}
\item adresa školy (obec, kód obce, ulice, ...)
\item právní informace (např. IČO, IZO, REDIZO, ...) 
\item informace o studiích (název školy, počet studentů, název oboru, ...) 
\item zeměpisné informace (souřadnice)
\item dodatečné informace (id školy v rámci databáze)\\
\end{itemize}

\subsection{Popis sloupců - adresa školy}
\todo{todo}

\subsection{Popis sloupců - právní informace}
V datasetu jsou zmíněny právní informace o škole, a to především jejich \textit{IČO}, \textit{IZO} a \textit{REDIZO}.\\
IČO znamená identifikační číslo osoby, což je osmímísné identifikační číslo právnické osoby. V datasetu se nachází IČO jako čtyřmístné až osmimístné. To může být zapříčiněno oříznutím všech nul nazačátku (může být takto uloženo v databázi nebo bylo upraveno při exportu) .V datasetu se nachází 54 odlišných IČO a nachází se ve všech záznamech.\\

IZO znamená identifikační znak organizace, který má formu devítimístného čísla a začíná číslicí 0, 1 nebo 2. IZO může být pro každou část školy různé (pokud jsou pod jednou hlavičkou dohromady střední škola, vyšší odborná škola, ...). V případě otevření datasetu pomocí excelu / openoffice / power BI se může IZO vyskytovat jako např. čtyřmístné číslo. To je zapříčiněno oříznutím všech nul, které se vyskytují nazačátku čísla. V některým případech může IZO být stejné jako IČO, jako v případě našeho datasetu. V datasetu se nachází 54 odlišných hodnot, které jsou buď stejné jako IČO, nebo začínají číslem 0 nebo 1 a mají 9 číslic.\\

REDIZO je resortní identifikátor právnické osoby, které je taktéž devitimístné číslo, ale zpravidla začínají číslicí 6 a je unikátní pro školu. V datasetu se nachází 54 odlišných hodnot, které splňují formát.



\subsection{Popis sloupců - informace o studiích}
V rámci tohoto bloku se popíše informace o oborech - název, typ, délka vzdělávání, počet studujících studentů, ...\\

První sloupec je \textit{obor\_nazev}, který značí název oboru na dané škole. Celkem se v datasetu vyskytuje 123 oborů.\\

Sloupec \textit{obor\_kod} označuje identifikační kód oboru, který se skládá z kombinace čísel a písmen a jeho délka je 7 znaků (ve zjednodušeném formátu bez pomlček a lomítka). Pro jeden obor může existovat více identifikačních kódu, které zároveň určují i druh vzdělávání  (např. gastronomie s odborným výcvikem a maturitou, gastronomie s vyučením i maturitou - nástavbové studium). Celkem se v datasetu vyskytuje 129 oborových kódů, což značí, že je zde 6 oborů, které mají rozdílný druh vzdělávání. Zároveň jsou zde špatně zadány 4 kódy - obsahují dodatečnou mezeru za posledním znakem. \\
\noindent Formát oborových kódů: \href{https://www.vyberskoly.cz/jak-se-vyznat-v-kodech-oboru}{https://www.vyberskoly.cz/jak-se-vyznat-v-kodech-oboru} \\

Sloupec \textit{vzdelavani\_delka\_roky} určuje délku studia oboru v počtu let. V datasetu existují obory, které lze studovat 1 rok (zkrácená studia nebo střední vzdělávání bez výučního listu a maturity), ale i osmiletá gymnázia. Žádná škola neposkytuje obor, který trvá 7 let. \\

Každý obor má i svůj druh vzdělávání, který je v datasetu reprezentován sloupcem \textit{vzdelavani\_druh}. Druh vzdělávání určuje výstup z daného oboru - maturita, výuční list, ... Celkem je zde 6 druhů vzdělávání:
\begin{itemize}
\item nástavbové studium - pro absolventy s výučním listem
\item střední vzdělávání
\item střední vzdělávání s maturitní zkouškou
\item střední vzdělávání s výučním listem
\item zkrácené studium k získání středního vzdělání s maturitní zkouškou
\item zkrácené studium k získání středního vzdělání s výučním listem\\
\end{itemize}

Další sloupec je \textit{vzdelavani\_forma}, který určuje formu vzdělávání oboru. V datasetu existují pouze 2 formy - denní a dálková. Dálková forma se ve většině případů vyskytuje hlavně u nástavbového studia. \\

Následující sloupce s názvem \textit{pocet\_studentu\_1\_rocnik} až \textit{pocet\_studentu\_8\_rocnik} vyjadřují počty studentů v ročnících. \\

Sloupec \textit{pocet\_studentu\_celkem} vyjadřuje celkový počet studentů ve všech ročnících na daném oboru. \\

Sloupec \textit{pocet\_absolventu\_2019\_2020} vyjadřuje počet absolventů daného oboru za školní rok 2019-2020. \\

Poslední sloupec \textit{pocet\_prijatych\_studentu} vyjadřuje celkový počet přijatých žáků do prvních ročníků daného oboru za tento rok. \\


\subsection{Popis sloupců - zeměpisné informace}
V datasetu se nachází sloupce popisující zeměpisné informace školy. Sloupce \textit{wkt}, \textit{x} a \textit{y} určují pozici školy na mapě.\\
Wkt je well-known text formát, který obsahuje souřadnice y,x a určuje pozici objektu na mapě. Tento sloupec nemá žádný informativní význam společně se sloupci \textit{x} a \textit{y} (adresa školy je detailně popsána, není potřeba ukládat souřadnice, alespoň ne v datasetu zkoumající počty studentů).


\subsection{Popis sloupců - dodatečné informace}
V datasetu se nachází sloupec s názvem \textit{OBJECTID}, které obsahuje jednotlivé číselné identifikátory oborů na školách v rámci databáze. Tento sloupec nemá žádný informativní význam.\\
Zárověň se zde nachází sloupec \textit{dp\_id}, u kterého nelze zjistit jakou nese informaci. Hodnoty typu 'ZOSS1' jsou nic neříkající a je zřejmé, že tento atribut nemá žádný informativní význam vzhledem k účelu tohoto datasetu. 



\newpage

\section{Použitelnost datasetu}
\todo{todo}
\noindent Atributy datasetu jsou rozděleny do 3 bloků na základě použitelnosti: významné, Nice-to-have a nepoužitelné\\
\noindent \textbf{Významné atributy}:
\begin{itemize}
\item 1 \\
\end{itemize}

\noindent \textbf{Nice-to-have atributy}:
\begin{itemize}
\item \textit{IČO} - rozhodně není potřeba ukládat do datasetu, ale pro některé lidi to může být zajímavá informace, díky které můžou o škole zjistit více informací - např. právní informace nebo více informací k oborům (za jde obor dobíhající, kapacita oboru, ...)
\item \textit{IZO} - stejné jako pro IČO
\item \textit{REDIZO} - stejné jako pro IČO \\
\end{itemize}

\noindent \textbf{Nepoužitelné atributy}:
\begin{itemize}
\item \textit{OBJECTID} - id oboru v interní databázi není potřeba ukládat do datasetů
\item \textit{wkt} - není potřeba ukládat zeměpisné informace o škole u datasetů zobrazující počty studentů na oborech, navíc v datasetu je uložena celá adresa školy (i jejich pracovišť)
\item \textit{x} - stejné jako pro wkt
\item \textit{y} - stejné jako pro wkt
\item \textit{dp\_id} - neidentifikovaltený atribut, který ale podle mého stejně nebude mít důležitou informaci k počtům studentů
\end{itemize}

\newpage

\section{Závěr}
\todo{todo}
počet škol, oborů
	
\end{document}